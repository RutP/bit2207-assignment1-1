\documentclass[12pt, a4paper]{report}
\usepackage[a4paper]{geometry}
\usepackage{lastpage}
\usepackage{graphicx, wrapfig, subcaption, setspace, booktabs}
\usepackage[T1]{fontenc}
\usepackage[font=small, labelfont=bf]{caption}
\usepackage{fourier}
\usepackage[protrusion=true, expansion=true]{microtype}
\usepackage[english]{babel}
\usepackage{sectsty}
\usepackage{hyperref}
\usepackage[table,xcdraw]{xcolor}

\onehalfspacing
\setcounter{tocdepth}{5}
\setcounter{secnumdepth}{5}

\begin{document}

\begin{titlepage}

\newcommand{\HRule}{\rule{\linewidth}{0.1mm}} % Defines a new command for the horizontal lines, change thickness here

\center % Center everything on the page

%----------------------------------------------------------------------------------------
%   MUK LOGO
%----------------------------------------------------------------------------------------

\begin{figure}
    \centering
  \includegraphics[width=0.4\textwidth]{muklogo.png}
\end{figure}
 
%----------------------------------------------------------------------------------------
%   HEADING SECTIONS
%----------------------------------------------------------------------------------------

\textsc{\LARGE Makerere University}\\[0.7cm] % Name of your university/college
\textsc{\Large CoCIS / SoCIT}\\[0.2cm] % Major heading such as course name
\textsc{\large B.Sc. Computer Science}\\[0.1cm] % Minor heading such as course title
\textsc{\small BIT 2207 Research Methodology}\\[0.8cm]

%----------------------------------------------------------------------------------------
%   TITLE SECTION
%----------------------------------------------------------------------------------------

\HRule \\[0.6cm]
\textsc{\Large Artificial intelligence and future of jobs for humans}\\[0.6cm]
\HRule \\[1.0cm]
 
%----------------------------------------------------------------------------------------
%   AUTHOR SECTION
%----------------------------------------------------------------------------------------

% \begin{minipage}{0.4\textwidth}
% \begin{flushleft} \large
% \emph{Author:}\\
% John \textsc{Smith} % Your name
% \end{flushleft}
% \end{minipage}
% ~
% \begin{minipage}{0.4\textwidth}
% \begin{flushright} \large
% \emph{Supervisor:} \\
% Dr. James \textsc{Smith} % Supervisor's Name
% \end{flushright}
% \end{minipage}\\[4cm]

\large
Harold \textsc{Turyasingura}\\
\textsc{10/u/11447/eve}\\[3cm] % Your name

{\large \today}

\vfill

\end{titlepage}

\tableofcontents
\newpage

\sectionfont{\scshape}
%----------------------------------------------------------------------------------------
%   SECTION
%----------------------------------------------------------------------------------------
\section*{Introduction}
\addcontentsline{toc}{section}{Introduction}
Jobs were previously a domain for only human employees. However, in recent years many
companies are investing in automation and use of Artificial Intelligence to replace human
workers with computer systems and robotic entities. Many factory jobs have been taken up
by robots, self-driving cars are being tested for both the transportation industry and for
personal use, and some computer systems are being employed in the financial and medical
industries to complement human workers. Jobs are now domain shared by both humans and
computer systems. But as computer systems get better and more efficient, humans might soon
find that they are no longer required to do any job. What will the future look like for
the human worker?\\

\subsection*{What is AI?}
\addcontentsline{toc}{subsection}{What is AI?}
Artificial Intelligence (AI) in the simplest terms is the ability of computer systems to mimic a
rational thought process. When we say that a computer system is Artificially Intelligent,
we mean that the system can derive independent conclusions when provided with a set of
data, which might not necessarily be organized. The task would be on the system to discover
patterns in the data, and use this information to make a decision.\\

The computer system could use concepts from Machine Learning, exploit a neural network,
to discover patterns and make decisions, then control robotic entities to complete a task.
AI can therefore be looked at as a combination of various fields of discipline
that enable computer systems to learn new skills, to make independent decisions and take action
on these decisions made.\\

In recent years, the growth of AI has been spurred by advancement in other areas of
technology, specifically in the increase of computational power available and at an ever decreasing
cost. This along with the drive by big corporate organizations (\emph{i.e.} Google, Facebook and
Microsoft among others) to join the Open Source Software movement, further increasing access to
quality AI starter kits like TensorFlow. All this has created an environment where
AI is not limited to corporate organizations but instead exposed to vast number of highly skilled
individuals who improve AI systems in various ways. This particularly has increased speed
of development of AI systems.\\

\subsection*{Current state of AI}
\addcontentsline{toc}{subsection}{Current state of AI}
AI today, is widely impressive, we have vehicles that are nearing the
dream of Self-Driving. Companies like Tesla have rolled out software updates to their vehicles
that enable them to drive without human intervention. George Hotz, an
independent developer, built and tested a plug a play system that would add self-driving capabilities
to any car, before he scrapped the project and open-sourced his code. IBM built Watson, an AI
entity that is being employed in the medical industry to assist doctors to make accurate diagnoses.\\

What does all this mean? Does it mean that in the not too distant future, we will not need
to drive our own cars, that we won't need factory workers or even doctors?
But how will humans survive in world where they cannot make a living?

\newpage
\section*{The rise of AI \emph{\textsc{(at the workplace)}}}
\addcontentsline{toc}{section}{The rise of AI (at the workplace)}
Companies are always searching for new ways to reduce costs, which in turn helps
to increase their profits. For some companies, employing a robotic workforce has
been found to cost less than an equivalent human workforce, not in upfront costs
but in long-term engagements. This is very attractive to many companies who like
to seek out the benefits of a robotic workforce. However this comes at a cost,
a loss of jobs for humans.\\

\end{document}